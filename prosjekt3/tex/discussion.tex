\section{Discussion}
\label{sec:discussion}

\subsection{Comparing Euler's and Verlet's method}
Comparing both efficiency and accuracy as described in the methods section, we see from Figure \ref{fig:compare euler verlet}
that the Forward Euler algorithm decreases in accuracy quite rapidly compared to the Velocity Verlet algorithm. This is most likely because VV conserves energy and angular momentum, as shown from our unit testing, where it remained conserved even for 200 years. From \ref{table:time} we see that FE is around twice as fast in all cases, which is to be expected owed to its less FLOPS. If we consider efficiency to be determined by the fastest runtime, it is clearly the most efficient. However, FE is so inaccurate that in the perspective that efficiency is accuracy per time, VV is the most efficient.

\subsection{Changing Newton's law of gravitation}
In Figure \ref{fig:changing beta} we see that decreasing the gravitational force between Earth and the Sun makes the system become unstable as $\beta \rightarrow 3$. Our perspective is that Earth in the true universe where $\beta = 2$ has a velocity that is comparable to the escape velocity in a universe where $\beta = 3$. We can see this as Earth's velocity seem to stabilise around zero after a while, corresponding to the definition of escape velocity, where the gravitational force approaches zero as does the velocity.


\subsection{Earth's escape velocity}
Experimentally changing Earth's velocity to find the escape velocity yields a result corresponding with our analytical expression described in \ref{sec:escvel} to a great degree. This then is an affirmation that our simple model behaves correctly.

\subsection{Stability when changing the mass of Jupiter}

Looking at Figure \ref{fig:jupiter mass} we see that Earth's orbit remains stable for a
ten-fold increase in Jupiter's mass. Increasing its mass a thousand-fold we see
that the Earth is evicted from its orbit around the sun, this is expected due to
Jupiter's already significant mass, and therefore effect on Earth's orbit. This also implies that in a hypothetical universe where Jupiter had this mass, a "Earth"-orbit wouldn't be stable due to Jupiter's influence, which could further imply that no planets could stay within the Goldilocks zone. In this solar system then, life would be far less likely based on current knowledge.

\subsection{Setting the centre of mass as the origin}
From Figure \ref{fig:centre of mass} we see that when setting the centre of mass as the origin, the radius of the orbit of the sun is about $5\cdot 10^{-3}$ AU. In comparison the radius of the sun is about $4.6\cdot 10^-{3}$ AU. So our modelled sun is still in the same area as the "real" sun should be, even when not correcting for the centre of mass. Considering that we model the astronomical objects as point masses, we already make a simplified model. So setting the sun as the origin is not a bad approximation. 

\subsection{Final model of the solar system}
As seen in Figure \ref{fig:solar system}, we see that the solar system has a rich
three dimensional structure, especially for the outer planets. Explaining this structure is outside the scope of this article, and would require a careful study of initial conditions of the solar system.


\subsection{The perihelion precession of Mercury}
Looking back at Figure \ref{fig:perihelion}, we can confirm that $43''$ of Mercury's perihelion
precession is accounted for by the general theory of relativity. Furthermore, the fact that without the correction, there
appears to be no precession. This is not the case in reality, as most of the precession is due to classical effects (see \ref{sec:rel}). In our model however, we only consider the Sun-Mercury system, and most of the classical precession effects are caused by interactions with other planets. It is therefore reasonable that we find the precession to be around zero for the classical case. The cause of this large effect from the theory of general relativity might be due to Mercury's proximity to the sun as it is more affected by the Sun's space-time curvature.

\subsection{Conclusion}
In conclusion, we find that object-oriented programming is very suitable for many-body systems as we were able to perform many different numerical experiments on a variety of planetary systems. As for our model, we obtain a lot of interesting results, from extreme cases like altering a law of nature to confirming our assumptions of a stationary sun. We find that substantially altering a parameter of the solar system can lead to its destabilisation. Newton's law of gravitation is not the whole story however, and using a general relativistic correction for Mercury's orbit around the Sun, we study the precession of its perihelion over a century, and find that our calculated $43''$ corresponds well with the reported anomaly from purely classical models.
