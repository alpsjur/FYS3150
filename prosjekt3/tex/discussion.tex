\section{Discussion}
\label{sec:discussion}

\subsection{Comparing Euler's and Verlet's method}

\subsection{Earth's escape velocity}
Consider then a planet which begins at a
distance of 1 AU from the sun. Find out by trial and error what the initial
velocity must be in order for the planet to escape from the sun. Can you find an
exact answer? How does that match your numerical results?
Try also to change the gravitional force, by replacing

where you let beta in [2, 3]. What happens to the earth-sun system when beta creeps
towards 3? Comment your results.


\subsection{Stability when changing the mass of Jupiter}

From Figure \ref{fig:jupiter mass} we see that Earth's orbit remains stable for a
ten-fold increase in Jupiter's mass. Increasing its mass a thousand-fold we see
that the Earth is evicted from its orbit around the sun, this is expected due to
Jupiter's already significant effect on Earth's orbit. This also implies that in a hypothetical universe where Jupiter had this mass, a "Earth"-orbit wouldn't be stable due to Jupiter's influence, which could further imply that no planets could stay within the Goldilocks zone. In this solar system then, life would be far less likely based on current knowledge.

\subsection{Setting the centre of mass as the origin}
From figure \ref{fig:centre of mass} we see that when setting the centre of mass as the origin, the radius of the orbit of the sun is about $5\cdot 10^{-3}$ AU. In comparison the radius of the sun is about $4.6\cdot 10^-3{}$ AU. Considering that we model the astronomical objects as point masses, when in reality they have a volume, setting the sun as
Finally,
using our Verlet solver, we carry out a real three-body calculation where all
three systems, the Earth, Jupiter and the Sun are in motion. To do this, choose
the center-of-mass position of the three-body system as the origin rather than
the position of the sun. Give the Sun an initial velocity which makes the total
momentum of the system exactly zero (the center-of-mass will remain fixed).
Compare these results with those from the previous exercise and comment your
results.

Considering again Figure \ref{fig:centre of mass}, we see that the Sun's orbit around
the centre of mass has a mean radius of much less than the radius of the Sun itself.
This implies that the centre of mass correction is very miniscule, and that our assumption
that the Sun is stationary seems to hold water for most practical purposes. We must be careful however, and distinguish between observing from the reference frame of the sun, and making the planets orbit the Sun. In the first case, the Sun is in the origin, but the planets orbit the centre of mass, while in the latter case the planets orbit the Sun and we don't consider the effect of a centre of mass. For extreme cases then, if we let the planets orbit the sun, we won't capture any possible effects on the Sun by changing planet parameters.

\subsection{Final model of the solar system}
Extend your program to include all planets in the solar system (if you
have time, you can also include the various moons, but it is not required) and
discuss your results.

As seen in Figure \ref{fig:solar system}, we see that the solar system has a rich
three dimensional structure, especially for the outer planets. Its configuration is determined by its
initial conditions way back then.


\subsection{The perihelion precession of Mercury}
Looking back at Figure \ref{fig:perihelion}, we can confirm that $43''$ of Mercury's perihelion
precession is accounted for by the general theory of relativity. Furthermore, the fact that without the correction, there
appears to be no precession. This is not the case in reality, as most of the precession is due to classical effects (see \ref{sec:rel}). In our model however, we only consider the Sun-Mercury system, and most of the classical precession effects are caused by interactions with other planets. It is therefore reasonable that we find the precession to be around zero for the classical case. The cause of this large effect from the theory of general relativity might be due to Mercury's proximity to the sun as it is more affected by the Sun's space-time curvature.
