\section{Discussion}
\label{sec:discussion}

\subsection{Comparing Euler's and Verlet's method}

\subsection{Earth's escape velocity}
Consider then a planet which begins at a
distance of 1 AU from the sun. Find out by trial and error what the initial
velocity must be in order for the planet to escape from the sun. Can you find an
exact answer? How does that match your numerical results?
Try also to change the gravitional force, by replacing

where you let beta in [2, 3]. What happens to the earth-sun system when beta creeps
towards 3? Comment your results.


\subsection{Stability when changing the mass of Jupiter}

Fra oppgaveteksten: Set up the algorithm and plot the positions of
the Earth and Jupiter using the velocity Verlet algorithm. Discuss the stability
of the solutions using your Verlet solver.
Repeat the calculations by increasing the mass of Jupiter by a factor of 10
and 1000 and plot the position of the Earth. Study again the stability of the
Verlet solver.

\subsection{Setting the centre of mass as the origin}
From figure \ref{fig:centre of mass} we see that when setting the centre of mass as the origin, the radius of the orbit of the sun is about $5\cdot 10^{-3}$ AU. In comparison the radius of the sun is about $4.6\cdot 10^-3{}$ AU. Considering that we model the astronomical objects as point masses, when in reality they have a volume, setting the sun as 
Finally,
using our Verlet solver, we carry out a real three-body calculation where all
three systems, the Earth, Jupiter and the Sun are in motion. To do this, choose
the center-of-mass position of the three-body system as the origin rather than
the position of the sun. Give the Sun an initial velocity which makes the total
momentum of the system exactly zero (the center-of-mass will remain fixed).
Compare these results with those from the previous exercise and comment your
results.

\subsection{Final model of the solar system}
Extend your program to include all planets in the solar system (if you
have time, you can also include the various moons, but it is not required) and
discuss your results.


\subsection{The perihelion precession of Mercury}
