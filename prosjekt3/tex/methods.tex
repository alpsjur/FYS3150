\section{Methods}
\label{sec:methods}

\subsection{Forces governing the motions in the solar system}
In this model we will consider only the gravitational forces between the bodies in the system. Newton's law of gravitation states that the force $\mathbf{F}_{ij}$ on a body $b_i$ with mass $M_i$ due to a body $b_j$ with mass $M_j$ at a distance $r_{ij}$ is given as
\begin{equation}\label{eq: gravity}
	\mathbf{F}_{ij} = \frac{GM_iM_j}{r_{ij}^2}\frac{\mathbf{r}_{ij}}{r_{ij}}.
\end{equation}
Here, $\mathbf{r}_{ij}$ is the vector between the centres of the bodies, pointing in the direction from $b_i$ towards $b_j$, and $G$ is the gravitational constant. Using Newton's second law we get the following differential equations for a system consisting of the bodies $b_1,b_2,\ldots,b_N$
\begin{equation}\label{eq:diff}
	\dv[2]{\mathbf{x}_i}{t} = \sum_j \frac{\mathbf{F}_{ij}}{M_i} = \sum_j \mathbf{a}_{ij}
\end{equation}
where $j = \{1,\ldots,N\}\setminus i$, and $\mathbf{x}_i$ is the position vector of body $b_i$.

Using astronomical units and years, we can, by approximating the orbit of Earth around the sun as a circle, find a scaled value for G. Setting $\mathbf{F}_{ij}$ in equation (\ref{eq: gravity}) equal to the centripetal force, we get
\begin{equation}\label{eq: centripetal}
	\frac{GM}{r^2} = \frac{Mv^2}{r}.
\end{equation}
Here, $M$ and $v$ are the mass and speed of Earth, respectively, and $r$ is the distance from the sun to Earth. By inserting the circular velocity into equation (\ref{eq: centripetal}), we get that
 \begin{align}
	 \frac{GM}{r} &= M\qty(\frac{2\pi r}{T})^2 = M\frac{4 \pi^2 r^2}{T^2} \\
	 G &= \frac{4 \pi^2 r^3}{T^2},
 \end{align}
where we set $r$ to be an AU and $T$ to be a year and get
 \begin{equation}
	 G = 4 \pi ^2 AU^3/yr^2.
 \end{equation}

\subsection{Numerical experiments}
\label{sec:numex}
In this section we outline the various numerical experiments we will perform on our
model solar system. One of them is looking at Jupiter's effect on Earth's orbit around the sun.
We will therefore see what happens if Jupiter's mass were to suddenly increase. The rest
of the experiments require a more careful description, and therefore get their own sections.
\subsubsection{Effects of the inverse square law}
\label{sec:beta}
	To understand how the form of NLG affects the motion of the planets, we will
	experiment by increasing the exponent of $r$. Naming this new exponent $\beta$,
	we have that a inverse beta law is given by
	\begin{equation}
		\label{eq:beta}
			F_G \propto \frac{1}{r^\beta}
	\end{equation}

\subsubsection{Escape velocity}
\label{sec:escvel}
One of the properties of the system we will solve for numerically is the escape velocity $v_e$ of a planet. This is a problem that has an analytical solution, which makes it a good test case. The escape velocity of an object $b_i$ orbiting $b_j$ is the velocity when the sum of the kinetic and potential energy is zero. That is
\begin{equation}
	\frac{1}{2}M_iv_e^2 = \frac{GM_iM_j}{r_{ij}}
\end{equation}
Letting $b_j$ be the sun, $r$ denote the distance of $b_i$ from the sun, and scaling the mass with $M_j$, we get
\begin{equation}
	\label{eq:esc}
	v_e = \sqrt{\frac{2G}{r}}
\end{equation}

\subsubsection{The relativistic case}
\label{sec:rel}
Newton's law of gravitation is not completely correct, and considered an approximation
to the theory of general relativity. This becomes especially apparent in the orbit of
Mercury, which is observed to precess around the sun. Specifically, the perihelion, the point where Mercury is furthest from the sun, precesses around the sun at around $575''$, where around $43''$ is
contributed from a general relativistic correction (see \cite{precession}, in particular table 3).
The correction to NGL is given by \ref{eq:relcor}
	\begin{equation}
		\label{eq:relcor}
		F_G = \frac{GM_{\text{sun}}m_{\text{mercury}}}{r^2}\qty(1 + \frac{3l^2}{r^2c^2}),
	\end{equation}
where $r$ is the distance between Mercury and the sun, $l$ is the angular momentum of
Mercury, and $c$ is the speed of light in vacuum. To study the effects of this
correction, we use the concept of a perihelion angle
	\begin{equation}
		\label{eq:perangle}
		\tan{\theta_p} = \frac{y_p}{x_y},
	\end{equation}
where $x_p$ and $y_p$ are the cartesian coordinates of Mercury at its perihelion.
Furthermore, we assume a perihelion speed and distance of $12.44\,$AU/yr and $0.3075\,$AU
respectively.

\subsection{Algorithms for solving the differential equations}
Consider the function
\begin{equation}
	x(t_k)=x_k,
\end{equation}
with known initial value $x_0$. We let
\begin{equation}
\left.\dv{x}{t}\right\rvert_{t_k}=v(t_k)=v_k.
\end{equation}
and
\begin{equation}
\left.\dv{v}{t}\right\rvert_{t_k} = a(t_k)=a_k.
\end{equation}
In the next two sections we will present two methods for approximating the values $x_{k}$, when the derivatives of $x$ or $v$ can be calculated. In Verlet's method both $x_k$ and $v_k$ are calculated using a known expression for the derivative of $v$, while in Euler's method the function $v$ must be known. We will however see that by using Euler's method twice, it is enough to have a known expression for $a$. The step size is given as $dt=\frac{t_n-t_1}{n}$.
\subsubsection{Euler's method}
\label{sec:fe}
Eulers method, also known as forward Euler, is based on a first order Taylor expansion around the point $t_k$. The function $x$ can then be written as
\begin{equation}
	x(t) = x_k + (t-t_i)v_k+\mathcal{O}\left(\left( t-t_k\right) ^2\right)
\end{equation}
Inserting $t=t_{k+1} = t_k + dt$, we get
\begin{equation}
	x_{k+1}=x_k + dtv_k + \mathcal{O}\left( dt^2\right)
\end{equation}
The algorithm then becomes
\begin{algorithm}[h!]
	\SetAlgoLined
		$x_{k+1} = x_k+dtv_k$\;
\end{algorithm}

In our case $a$ is the only known function. By using the algorithm above twice, we can first compute $v_k$ and then $x_k$. Our modified algorithm is
\begin{algorithm}[h!]
	\SetAlgoLined
		compute $a_k$\;
		$v_{k+1} = v_k+dta_k$\;
		$x_{k+1} = x_k+dtv_k$\;
\end{algorithm}

\subsubsection{Velocity Verlet}
\label{sec:vv}
We now consider the second order Taylor expansion of $x$ and $v$ around the point $t_k$. Then we can write $x_{k+1}$ and $v_{k+1}$ as
\begin{align} \label{eq:vvy}
	x_{k+1}&=x_k + dtv_k + \frac{dt^2}{2}a_k  +\mathcal{O}(dt^3) \\
	v_{k+1}&=v_k + dta_k + \frac{dt^2}{2}\left.\dv{a}{t}\right\rvert_{t_k}  +\mathcal{O}(dt^3)
\end{align}
The derivative of $a$ can be approximated as
\begin{equation}
	 \left.\dv{a}{t}\right\rvert_{t_k} \approx \frac{a_{k+1}-a_k}{dt}
\end{equation}
which gives us the following expression for $v_{k+1}$
\begin{equation}\label{eq:vvf}
	v_{k+1}=v_k + \frac{dt}{2}\left(a_{k+1} + a_k \right)  +\mathcal{O}(dt^3)
\end{equation}
Combining equation (\ref{eq:vvy}) and (\ref{eq:vvf}) gives us the following algorithm
\begin{algorithm}[h!]
	\SetAlgoLined
		compute $a_k$\;
		$x_{k+1} = x_k+dtv_k+\frac{dt^2}{2}g_k$\;
		compute $a_{k+1}$\;
		$v_{k+1} = v_k + \frac{dt}{2}\left(a_{k+1}+a_k\right)$\;
\end{algorithm}

\subsubsection{Solving for all bodies}
Let's again look at equation (\ref{eq:diff}). We let
\begin{equation}
\mathbf{x}_i(t_k)=\mathbf{x}_{i}^k
\end{equation}
and
\begin{equation}
\left.\dv[2]{\mathbf{x}_i}{t}\right\rvert_{t_k} = \mathbf{a}_i(t_k)=\mathbf{a}_{i}^k
\end{equation}
where $k=1,2,\ldots,n$. The algorithm for computing the orbits of all the bodies in the system can be written as
\begin{algorithm}[h!]
	\SetAlgoLined
	\For{$k=1,\ldots,n$ }{
		\For{$i=1,\ldots,N$}{
			set $\mathbf{a}_{i}^k=0$\;
			\For{$j=1,\ldots,N$}{
				\If{$i\neq j$}{
					$\mathbf{a}_{i}^k \mathrel{+}= \mathbf{a}_{ij}$\;
				}
			}
			compute $\mathbf{x}_{i}^k$\;
		}
	}
\end{algorithm}
where $\mathbf{x}_{i}^k$ is computed using either Euler's method or Verlet's method.

\subsection{Implementation}
All our programs are written in C++, using python3.6 to produce figures and tables.
We make use of vectors and strings from the standard library, as well as boost,
which makes creating directories simpler. All our results are reproducible
by running a main python program.

We create three classes: the Coordinate class to handle vector arithmetics, the Planet
class which creates a object with a name, mass and unallocated position and velocity Coordinate vectors,
and the final class, our System class which takes Planet objects. Initialising
a vector of Planet objects, and solving for their motion in a gravitational field, the System
class will be our solver paradigm. In this solver, we implement both the Forward Euler and
Velocity Verlet algorithm, as well as methods for getting the total energy and angular
momentum, as well as varying different parameters like r dependence of NGL and initial velocity.
We also allow for a general relativistic correction to NGL, as decribed in Section \ref{sec:rel}

All our code can be found in
a github repository "FYS3150" by janadr\footnote{https://github.com/janadr/FYS3150/tree/master/prosjekt3}.
\subsubsection{Comparing methods and unit testing} \label{sec:tests}
In our model of the solar system, both angular momentum and mechanical energy should be conserved, as there
are no external forces or torques acting on the system. This means that at any point in time, the total energy or
angular momentum of the system should be the same as initially. An ideal method should conserve these methods,
and therefore we test both algorithms for energy and angular momentum conservation.
Furthermore, we implement this as a unit test for the VV algorithm, as we expect this method to conserve these quantities.

Another simple comparison is to time the two algorithms in order to determine which is the most efficient. When timing the algorithms we run both algorithms with the same input parameters X times, and record the mean time and standard deviation.
We also check if Earth escapes its orbit if we set its initial velocity to be the analytical
escape velocity given by \ref{eq:esc}.
One more test that could be implemented, but which we will not, is comparing Venus' 2D orbit with a circle
with radius the mean distance of Venus from the sun. As we expect this orbit to be
almost circular due to its low eccentricity \cite{planetaryfactsheet}. This would then
be a way to quantify our error, which we could use to compare the algorithms.

\subsubsection{Obtaining initial values}
The initial velocities and positions for the planets, together with the masses, were collected from NASA's web-based interface to JPL's HORIZONS system. For further references, see \cite{horizon}. October 5th 2018 was used as the initial time.
