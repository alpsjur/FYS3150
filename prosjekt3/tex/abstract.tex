\begin{abstract}
  We develop an object-oriented program for simulating many-body systems. In particular,
  we create a model of the solar system. First, we compare two different numerical methods of solving ordinary differential equations: the Forward Euler (FE) and Velocity Verlet (VV) algorithms. Considering efficiency and accuracy,
  we find that the VV algorithm is clearly superior in accuracy as it conserves energy and angular momentum. While FE runs faster by about $number$ when comparing integration steps, its low accuracy makes VV more efficient in practice. Performing various numerical experiments on different configurations of planets, we consider the Sun-Earth-(Jupiter) and Sun-Mercury (SE, SEJ, SM) systems. In the SE system we look at the effects of altering Newton's law of gravitation (NLG). We find that changing NLG's
  $r$-dependence to $r^3$ lets Earth escape its orbit, destabilising the system. As for the SEJ system, we increase Jupiter's mass again destabilising Earth's orbit. We consider then the effect of the centre of mass on Earth and Jupiter's orbits, and find that it is for most purposes negligible. Finally, studying the SM system, specifically Mercury's perihelion precession, we found a total of 43 arcseconds contributed from a general relativistic correction, corresponding very well to the observed anomaly from purely classical models.
\end{abstract}
