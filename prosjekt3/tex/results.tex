\section{Results}
\label{sec:results}



In Figure \ref{fig:earth escape} we see that
\begin{figure}[htbp]
	\centering
%	\includegraphics[width=0.5\textwidth]{filnavn.pdf}
	\caption{The trajectory of Earth when setting the initial velocity to the escape velocity.}
	\label{fig:earth escape}
\end{figure}

In Figure \ref{fig:changing beta} we show the distance and velocity of the Earth from the sun, and the effects of increasing the exponent in the $\frac{1}{r^2}\,$-proportionality of NLG as described in Section \ref{sec:beta}. We see that for the case where $\beta=2.5$ the phase of the distance and velocity seem to shift toward the right, with perhaps a slight increase in amplitude for both values. Increasing $\beta$ to the $[2.9, 3]$ range we see an increase in amplitude for $\beta = 2.9$ and a greater wavelength for both values. As $\beta \rightarrow 3\,$, the speed rapidly increases while the distance decreases.
\begin{figure}[htbp]
	\centering
	\includegraphics[width=0.5\textwidth]{change_beta_10yr.pdf}
	\caption{The speed and distance, from the sun, of Earth when changing the gravitational force. The $\beta$ dependence is shown in Eq. \ref{eq:beta}.}
	\label{fig:changing beta}
\end{figure}

Increasing Jupiter's mass as described in Section \ref{sec:numex}, we produce Figure \ref{fig:jupiter mass}. Here the original orbits are shown as reference. Increasing Jupiter's mass tenfold, we see that Earth's orbit is shifted outward towards Jupiter.
For $1000m_{\text{Jupiter}}$, it appears that Earth is dislodged from its orbit around the sun, while a sun-Jupiter interaction seems to have come into effect.
\begin{figure}[htbp]
	\centering
	\includegraphics[width=0.5\textwidth]{jupiter_mass.pdf}
	\caption{Plot showing the trajectories of Earth, Jupiter and the Sun for different values for the mass of Jupiter. The upper plot shows the original orbits of a Sun-Earth-Jupiter system, i. e. $1m_{\text{Jupiter}}$. On the bottom left is the orbits when $10m_{\text{Jupiter}}$, and on the bottom right when $1000m_{\text{Jupiter}}$.}
	\label{fig:jupiter mass}
\end{figure}

Shown in Figure \ref{fig:centre of mass} is the motion of the Sun-Earth-Jupiter system.
On the left we show the centre of mass reference frame with the sun's motion at the bottom, while on
the right we show the Sun's reference frame, with the motion of the centre of mass at the bottom.
We see that there is no great change in the planets' orbits comparing the two. Furthermore,
we see that the Sun orbits the centre of mass in a circle. The motion of the centre of mass itself, looks to periodically move in a half-ellipse.
\begin{figure}[htbp]
	\centering
	\includegraphics[width=0.5\textwidth]{center_of_mass.pdf}
	\caption{The orbits of Earth, Jupiter and the sun with sun as the initial origin to the right and the centre of mass as the initial origin to the left.}
	\label{fig:centre of mass}
\end{figure}

Figure \ref{fig:solar system} shows the 3D-structure of the solar system
\begin{figure}[htbp]
	\centering
	\includegraphics[width=0.5\textwidth]{solarsystem3d.pdf}
	\caption{The model of the solar system including all the planets and setting the centre of mass as the origin.}
	\label{fig:solar system}
\end{figure}
