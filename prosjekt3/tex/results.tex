\section{Results}
\label{sec:results}

Table \ref{table:time} shows the ratio between the CPU-time for the Velocity Verlet and the Forward Euler algorithm. We see that for all end times T Forward Euler is about twice as efficient. 

\begin{table}[htbp]
	\centering
	\begin{tabular}{lrr}
		T   & ratio & $\sigma$     \\
		\hline
		\addlinespace[0.1cm]
		1   & 1.83      & 5.56e-2 \\
		10  & 1.96      & 2.23e-2 \\
		100 & 1.98      & 2.08e-2 \\
		500 & 1.97      & 3.21e-2
	\end{tabular} \caption{The ratio between the CPU time for the VV and the FE algorithm for different end times T. The mean and standard deviation ($\sigma$) of 10 runs is recorded.}\label{table:time}
\end{table}

Considering Figure \ref{fig:compare euler verlet}, we see that the Earth's orbit is not regular over time when Euler's method is used. The Velocity Verlet method gives a non-changing orbit over time. 
\begin{figure}[htbp]
	\centering
	\includegraphics[width=0.5\textwidth]{eulerVerlet.pdf}
	\caption{Graphical comparison of the Forward Euler and Velocity verlet algorithms for $\Delta t = 0.001$. The upper one is for T=1, lower left T=5 and lower right t=10.}
	\label{fig:compare euler verlet}
\end{figure}


In Figure \ref{fig:earth escape} we see that Earth escapes from the gravitational field of the sun when the velocity is set to the value calculated from equation (\ref{eq:esc}). Earth does not escape for a velocity set to 90\% of the escape velocity.  
\begin{figure}[htbp]
	\centering
	\includegraphics[width=0.5\textwidth]{escape.pdf}
	\caption{The trajectory of Earth when setting the initial velocity to the escape velocity, and to 90\% of the velocity. The origin is set to the sun.}
	\label{fig:earth escape}
\end{figure}

In Figure \ref{fig:changing beta} we show the speed of the Earth and the distance of the Earth from the sun, and how these values are affected by increasing the exponent in the $\frac{1}{r^2}\,$-proportionality of NLG as described in Section \ref{sec:beta}. We see that for the case where $\beta=2.5$ the phase of the distance and velocity seem to shift toward the right, with perhaps a slight increase in amplitude for both values. Increasing $\beta$ to the $[2.9, 3]$ range we see an increase in amplitude for $\beta = 2.9$ and a greater wavelength for both values. As $\beta \rightarrow 3\,$, the speed rapidly increases while the distance decreases. Then, after about 3 yr for $\beta = 2.9$ and 6 yr for $\beta = 3$, the distance increases while the speed decreases. The distance continue to increase, as can be seen in Figure \ref{fig:changing beta60} in the appendix.
\begin{figure}[htbp]
	\centering
	\includegraphics[width=0.5\textwidth]{change_beta_10yr.pdf}
	\caption{The speed and distance, from the sun, of Earth when changing the gravitational force. The $\beta$ dependence is shown in Eq. \ref{eq:beta}.}
	\label{fig:changing beta}
\end{figure}

Increasing Jupiter's mass as described in Section \ref{sec:numex}, we produced Figure \ref{fig:jupiter mass}. Here the original orbits are shown as reference. Increasing Jupiter's mass tenfold, we saw that the orbits remained the same.
For $1000m_{\text{Jupiter}}$, it appears that Earth is dislodged from its orbit around the sun, while Jupiter's orbit is more elliptical and closer to the sun.
\begin{figure}[htbp]
	\centering
	\includegraphics[width=0.5\textwidth]{jupiter_mass.pdf}
	\caption{Plot showing the trajectories of Earth, Jupiter and the sun for different values for the mass of Jupiter. The upper plot shows the original orbits of a Sun-Earth-Jupiter system, i. e. $1m_{\text{Jupiter}}$. On the bottom left is the orbits when $10m_{\text{Jupiter}}$, and on the bottom right when $1000m_{\text{Jupiter}}$.}
	\label{fig:jupiter mass}
\end{figure}

Looking now at Figure \ref{fig:centre of mass}, we show the orbit of the sun when the centre of mass is set as the origin and the sun is given an initial velocity so that the momentum of the system is zero. The figure also shows how the absolute value of the position vector of Jupiter and Earth changes when the origin is changed from the sun to the centre of mass. Both the radius of Sun's orbit and the maximum change in the position vector of Earth has a magnitude of about $10^{-3}$ AU. The change in the position vector of Jupiter has a maximum magnitude of about $10^{-2}$ AU.
\begin{figure}[htbp]
	\centering
	\includegraphics[width=0.5\textwidth]{center_of_mass.pdf}
	\caption{The upper figure shows the orbit of the sun when the centre of mass is set as the origin. The lower figure shows the difference between the distance to the centre of mass and the distance to the sun for Jupiter and Earth.}
	\label{fig:centre of mass}
\end{figure}

In Figure \ref{fig:changing beta} the final model including all the planets is shown in three dimensions. For a two dimensional plot of the xy-plane, see Figure \ref{fig:solar system2d} in the Appendix.
\begin{figure}[htbp]
	\centering
	\includegraphics[width=0.5\textwidth]{solarsystem3d.pdf}
	\caption{The model of the solar system including all the planets and setting the centre of mass as the origin.}
	\label{fig:solar system}
\end{figure}

Figure \ref{fig:perihelion} shows the perihelion precession of Mercury over a century. Both the classical and the relativistic case is shown. We see that in the relativistic case Mercury has a linear change in precession of about 43 arc seconds, while in the classical case the change is close to zero.
\begin{figure}[htbp]
	\centering
	\includegraphics[width=0.5\textwidth]{perihelion.pdf}
	\caption{The perihelion precession of Mercury both with and without the relativistic correction. $dt$ is set to $10^{-7}$ AU.}
	\label{fig:perihelion}
\end{figure}
