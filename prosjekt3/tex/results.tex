\section{Results}
\label{sec:results}



In Figure \ref{fig:earth escape} we see that
\begin{figure}[htbp]
	\centering
%	\includegraphics[width=0.5\textwidth]{filnavn.pdf}
	\caption{The trajectory of Earth when setting the initial velocity to the escape velocity.}
	\label{fig:earth escape}
\end{figure}

In Figure \ref{fig:changing beta} we observe the effects of increasing the exponent in the $\frac{1}{r^2}\,$-proportionality of NLG as described in Section \ref{sec:beta}.
\begin{figure}[htbp]
	\centering
	\includegraphics[width=0.5\textwidth]{change_beta_10yr.pdf}
	\caption{The speed and distance, from the sun, of Earth when changing the gravitational force. The $\beta$ dependence is shown in Eq. \ref{eq:beta}.}
	\label{fig:changing beta}
\end{figure}

\begin{figure}[htbp]
	\centering
	\includegraphics[width=0.5\textwidth]{jupiter_mass.pdf}
	\caption{Plot showing the trajectories of Earth, Jupiter and the sun for different values for the mass of Jupiter.}
	\label{fig:jupiter mass}
\end{figure}

\begin{figure}[htbp]
	\centering
	\includegraphics[width=0.5\textwidth]{center_of_mass.pdf}
	\caption{The orbits of Earth, Jupiter and the sun with sun as the initial origin to the right and the centre of mass as the initial origin to the left.}
	\label{fig:centre of mass}
\end{figure}

\begin{figure}[htbp]
	\centering
	\includegraphics[width=0.5\textwidth]{solarsystem3d.pdf}
	\caption{The model of the solar system including all the planets and setting the centre of mass as the origin.}
	\label{fig:solar system}
\end{figure}
