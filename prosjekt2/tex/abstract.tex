\begin{abstract}
  We implement Jacobi's method for solving eigenvalue problems on the form $A\mathbf{v} = \lambda \mathbf{v}$, where $A$ is a symmetric matrix. By scaling the equations, we can use our algorithm to solve the buckling beam, single electron in an harmonic oscillator potential, and two electrons in an harmonic oscillator potential problems. By looking at the buckling beam problem, we find that the number of iterations needed to diagonalize the matrix is proportional to $n^2$, while the CPU-time is proportional to $n^4$. The relative error in the eigenvalues converges to $\sim10^{-5.77}$ for $n$ larger than 25. For $n$ larger than 20, the Armadillo function \texttt{eig\_sym} is faster than Jacobi's method. We therefore conclude that Jacobi's method gives a good approximation, even though it is slow for large $n$. We also studied the case of the two electrons
  in more detail, and found that a stronger potential results in a narrower wave function and higher energy eigenvalues.
  Furthermore, we found that the Coloumb interaction acts to increase the distance between the two electrons,
  but not necessarily increase the spread in the wave function. This is because of the new equillibrium obtained
  between the harmonic oscillator potential and the Coloumb interaction, and we also see this in the increasing
  increase of the energy eigenvalues as a function of $\omega_r$. 
\end{abstract}
