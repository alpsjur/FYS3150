\onecolumn
\setcounter{equation}{0}
\renewcommand\theequation{A.\arabic{equation}}
\section*{A}
\label{sec:appendix}

  \[examplematrix =
    \mqty[b_1 & c_1 & 0 & \hdots & \hdots & 0 \\
          a_1 & b_2 & c_2 & 0 & \hdots & 0 \\
          0 & a_2 & b_3 & c_3 & \hdots & 0 \\
          \vdots & \ddots & \ddots & \ddots & \ddots & \vdots \\
          0 & \hdots & \ddots & a_{n-2} & b_{n-1} & c_{n-1} \\
          0 & \hdots & \hdots & 0 & a_{n-1} & b_n],
  \]

\subsection*{Derivation of the radial equation}
\label{sec:radeq}
Considering the case of a particle in a three dimensional harmonic oscillator
potential, we have the Hamiltonian
  \begin{equation}
    H = -\frac{\hbar^2}{2m}\laplacian + V\qty(r), \quad r = \sqrt{x^2 + y^2 + z^2}
  \end{equation}
which in spherical coordinates is given by
  \begin{equation}
    H = -\frac{\hbar^2}{2m}\frac{1}{r^2}\pdv{}{r}\qty(r^2\pdv{}{r}) + V\qty(r) + \frac{L^2}{2mr^2},
  \end{equation}
where
  \begin{equation}
    L^2 = -\hbar^2 \qty[\frac{1}{\sin\theta}\pdv{}{\theta}\qty(\sin{\theta}\pdv{\theta})
    + \frac{1}{\sin^2\theta}\pdv[2]{}{\phi}],
  \end{equation}
is the squared angular momentum operator. Since $L^2$ does not act on
$r$, we have that $H$ and $L^2$ commute, and neglecting spin, a quantum state can
then be represented as a wavefunction
  \begin{equation}
    \ket{\Psi} \simeq \Psi\qty(r, \phi, \theta) = R\qty(r)Y_{l}^m\qty(\phi, \theta),
  \end{equation}
where $R\qty(r)$ is the radial solution, and $Y_{l}^m\qty(\theta, \phi)$ are the
eigenfunctions of $L^2$ (spherical harmonics).
The Schrodinger equation is then
  \begin{equation}
    HR\qty(r)Y_{l}^m\qty(\phi, \theta) = ER\qty(r)Y_{l}^m\qty(\phi, \theta).
  \end{equation}
Here $E$ are the eigenvalues of $\Psi\qty(r, \theta, \phi)$. Letting $L^2$ act on $Y_{m}^l\qty(\theta, \phi)$ we obtain
its eigenvalues $\hbar^2 l\qty(l+1)$. Substituting $R\qty(r) = \frac{u\qty(r)}{r}$,
we see that
  \[
    \frac{1}{r^2}\pdv{}{r}\qty(r^2\pdv{}{r})\frac{u\qty(r)}{r}
    = \frac{1}{r^2}\pdv{}{r}\qty[r^2\qty(\pdv{u}{r}\frac{1}{r} - \frac{u\qty(r)}{r^2})]
    = \frac{1}{r^2}\qty(\pdv{u}{r} + r\pdv[2]{u}{r} - \pdv{u}{r}) = \frac{1}{r}\pdv[2]{u}{r}.
  \]
Multiplying by r on both sides, the problem then reduces to the radial equation
\begin{equation}
  -\frac{\hbar^2}{2m}\qty(\dv[2]{}{r} - V(r) - \frac{l\qty(l+1)}{r^2})u\qty(r) = Eu\qty(r).
\end{equation}

\subsection*{Expressions for computing the Jacobi rotation}
 The resulting matrix from the transformation in eq. \ref{eq:sim}, written on the same form as eq. \ref{eq:S}, is 
 \begin{equation}\label{eq:rotate}
 \begin{split}
 &b_{ii} = a_{ii}, \quad i\neq k,l  \\
 &b_{ik} = a_{ik}\cos{\theta} - a_{il}\sin \theta, \quad i\neq k,l \\
 &b_{il} = a_{il}\cos{\theta} + a_{ik}\sin \theta, \quad i\neq k,l \\
 &b_{kk} = a_{ll}\cos^2{\theta} - 2a_{kl}\cos{\theta}\sin{\theta} + a_{kk}\sin^2{\theta}  \\
 &b_{ll} = a_{kk}\cos^2{\theta} + 2a_{kl}\cos{\theta}\sin{\theta} + a_{ll}\sin^2{\theta}  \\
 &b_{kl} = \left( a_{kk} - a_{ll}\right) \cos{\theta}\sin{\theta} + a_{kl}\left( \cos^2\theta - \sin^2\theta\right) 
 \end{split}
 \end{equation}
 Using eq. \ref{eq:rotate}, and letting $\tau = \cot{2\theta}$, we get that to satisfy $b_{kl}=0$, we must have 
 \begin{equation}
  \tau =  \frac{a_{ll}-a_{kk}}{2a_{kl}} 
 \end{equation}
 By using the trigonometric identity $\cot{2\theta} = \frac{1}{2}\left( \cot \theta - \tan \theta\right)$, we get the quadratic equation
 \begin{equation}
 \tan^2 \theta + 2\tau \tan \theta - 1 = 0
 \end{equation}
 with the solution
 \begin{equation}
 \begin{split}
 &\tan \theta = -\tau \pm \sqrt{1+\tau^2} \\
 &\cos \theta = \frac{1}{\sqrt{1+\tan^2 \theta}} \\
 &\sin \theta = \tan \theta \cos \theta 
 \end{split}
 \end{equation}
