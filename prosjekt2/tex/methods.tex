\section{Methods}
\label{sec:methods}

\subsection{The buckling beam}
\label{sec:bucklingbeam}

Considering first the buckling beam problem, we have
  \begin{equation}
    \gamma \dv[2]{u}{x} = -F u\qty(x), \quad x \in [0, L]
  \end{equation}
where $\gamma$ is a property constant, $u\qty(x)$ the vertical displacement, and
$F$ the force applied at (L, 0) towards the origin. We can scale this equation by defining
a parameter $\rho = \frac{x}{L}$. Inserting, we get
  \begin{equation}
    -\dv[2]{}{\rho}u\qty(\rho) = \lambda u\qty(\rho), \quad \rho \in [0, 1]
  \end{equation}
where $\lambda = \frac{FL^2}{\gamma}$. Now we see that this equation is on the form
of eq. \ref{eq:eigenspecial}. However, enforcing Dirichlet boundary conditions
$u\qty(0) = u\qty(1) = 0$ and using a 2nd order central approximation for n integration steps:
  \begin{equation}
  \label{eq:disc1}
    \frac{v_{i+1} - 2v_i + v_{i-1}}{h^2} + \order{h^2} = \lambda_i v_i,
  \end{equation}
where $h = \frac{1}{n+1}$. Disregarding the boundaries (which are set to 0) we obtain the eigenvalue equations
  \begin{equation}
  \label{eq:buckbeam}
    A\vb{v} = \lambda \vb{v}.
  \end{equation}
Here
  \begin{equation}
    A =
      \mqty[d & a & 0 & \hdots & \hdots & 0 \\
            a & d & a & 0 & \hdots & 0 \\
            0 & a & d & a & \hdots & 0 \\
            \vdots & \ddots & \ddots & \ddots & \ddots & \vdots \\
            0 & \hdots & \ddots & a & d & a \\
            0 & \hdots & \hdots & 0 & a & d],
  \end{equation}
 is an tridiagonal matrix where $d = \frac{2}{h^2}$ and $a = -\frac{1}{h^2}$,
 $\lambda$ is an eigenvalue, and $\vb{v} \in \qty(0, n)$ is an eigenvector.
 The analytical eigenvalues are given by
  \begin{equation}
    \lambda_j = d + 2a\cos\qty(\frac{j\pi}{n + 1}),
  \end{equation}
$j = 1, 2, \dots n-1$.



\subsection{Quantum dots}
\label{sec:qmdots}

We want to model an electron in a three dimensional harmonic oscillator potential
  \begin{equation}
    V\qty(r) = \frac{1}{2}m\omega^2r^2, \quad r=\sqrt{x^2 + y^2 + z^2}
  \end{equation}
$r \in \qty(0, \infty)$, $m$ is the mass, and $\omega$ is the frequency.
The quantum state can then be represented as the wavefunction
\begin{equation}
  \ket{\Psi} \simeq \Psi\qty(r, \phi, \theta) = R\qty(r)Y_{l}^m\qty(\phi, \theta),
\end{equation}
where $R\qty(r)$ is the radial part, and $Y_{l}^m\qty(\theta, \phi)$ are the spherical
harmonics. For reasons that will be elaborated further later, what we need to solve
then is the radial equation (see appendix for more details on the wavefunction and the
radial eq.)
\begin{equation}
  \label{eq:radeq}
  -\frac{\hbar^2}{2m}\qty(\dv[2]{}{r} - V\qty(r) - \frac{l\qty(l+1)}{r^2})u\qty(r) = Eu\qty(r).
\end{equation}
Here $l$ is the orbital momentum, $u\qty(r) = rR\qty(r)$, and $E$ are the
eigenvalues of $\Psi\qty(r, \theta, \phi)$.
We will assume our electron has no orbital momentum ($l = 0$), and scale eq. \ref{eq:radeq}
by substituting $\rho = \frac{r}{\alpha}$ and $\lambda = \frac{E}{\epsilon}$, inserting $V\qty(r)$, and get
  \begin{equation}
    -\frac{\hbar^2}{2m\alpha^2}\qty(\dv[2]{}{\rho} - \frac{m^2\omega^2 \alpha^4}{\hbar^2} \rho^2)u\qty(\rho) = \epsilon\lambda u\qty(\rho),
  \end{equation}
where we can define a natural energy scale $\epsilon = \frac{\hbar^2}{2m\alpha^2}$,
and a natural length scale $\alpha = \sqrt{\frac{\hbar}{m\omega}}$, yielding the
dimensionless equation
  \begin{equation}
    -\dv[2]{u}{\rho} + \rho^2u\qty(\rho) = \lambda u\qty(\rho).
  \end{equation}
Discretising as in eq. \ref{eq:disc1}, the equation becomes
  \begin{equation}
  \label{eq:disc2}
    -\frac{v_{i+1} - 2v_i + v_{i-1}}{h^2} + V_i v_i = \lambda v_i,
  \end{equation}
where $V_i = \rho_{i}^2=\qty(ih)^2$. Enforcing the Dirichlet boundary conditions,
we see that \ref{eq:disc2} can be written on the same form as \ref{eq:buckbeam}
  \begin{equation}
    A\vb{v} = \lambda \vb{v}.
  \end{equation}
Here
  \begin{equation}
    A =
      \mqty[\widetilde{d}_1 & a & 0 & \hdots & \hdots & 0 \\
            a & \widetilde{d}_2 & a & 0 & \hdots & 0 \\
            0 & a & \widetilde{d}_3 & a & \hdots & 0 \\
            \vdots & \ddots & \ddots & \ddots & \ddots & \vdots \\
            0 & \hdots & \ddots & a & \widetilde{d}_{n-1} & a \\
            0 & \hdots & \hdots & 0 & a & \widetilde{d}_n],
  \end{equation}
where $\widetilde{d_i} = \frac{2}{h^2} + V_i$ and $a$ is as before.

\subsection{Algorithm: Jacobi's determinant}
\label{sec:jacobi}

\subsection{Programming technicalities}
\label{sec:progtech}
