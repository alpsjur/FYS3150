\section{Methods}
\label{sec:methods}

We will FLAGG

\subsection{The buckling beam}
\label{sec:bucklingbeam}

Considering first the buckling beam problem, we have
  \begin{equation}
    \gamma \dv[2]{u}{x} = -F u\qty(x), \quad x \in [0, L]
  \end{equation}
where $\gamma$ is a property constant, $u\qty(x)$ the vertical displacement, and
$F$ the force applied at (L, 0) towards the origin. We can scale this equation by defining
a parameter $\rho = \frac{x}{L}$. Inserting, we get
  \begin{equation}
    -\dv[2]{}{\rho}u\qty(\rho) = \lambda u\qty(\rho), \quad \rho \in [0, 1]
  \end{equation}
where $\lambda = \frac{FL^2}{\gamma}$. Now we see that this equation is on the form
of eq. \ref{eq:eigenspecial}. However, enforcing Dirichlet boundary conditions
$u\qty(0) = u\qty(1) = 0$ and using a 2nd order central approximation for n integration steps we obtain
  \begin{equation}
  \label{eq:disc1}
    -\frac{v_{i+1} - 2v_i + v_{i-1}}{h^2} + \order{h^2} = \lambda_i v_i,
  \end{equation}
where $h = \frac{\rho_n - rho_0}{n}$. Disregarding the boundaries (which are set to 0) we obtain the eigenvalue equation
  \begin{equation}
  \label{eq:buckbeam}
    A\vb{v} = \lambda \vb{v}.
  \end{equation}
Here
  \begin{equation}
    A =
      \mqty[d & a & 0 & \hdots & \hdots & 0 \\
            a & d & a & 0 & \hdots & 0 \\
            0 & a & d & a & \hdots & 0 \\
            \vdots & \ddots & \ddots & \ddots & \ddots & \vdots \\
            0 & \hdots & \ddots & a & d & a \\
            0 & \hdots & \hdots & 0 & a & d],
  \end{equation}
 is an tridiagonal matrix where $d = \frac{2}{h^2}$ and $a = -\frac{1}{h^2}$,
 $\lambda$ is an eigenvalue, and $\vb{v} \in \qty(0, n)$ is an eigenvector.
 The analytical eigenvalues are given by
  \begin{equation}
    \lambda_j = d + 2a\cos\qty(\frac{j\pi}{n + 1}),
  \end{equation}
$j = 1, 2, \dots n-1$.



\subsection{Single electron in an harmonic oscillator potential}
\label{sec:qmdot}

We want to model an electron in a three dimensional harmonic oscillator potential
  \begin{equation}
    V\qty(r) = \frac{1}{2}m\omega^2r^2, \quad r=\sqrt{x^2 + y^2 + z^2}
  \end{equation}
$r \in \qty(0, \infty)$ FLAGG, $m$ is the mass, and $\omega$ is the frequency.
The quantum state can then be represented as the wavefunction
\begin{equation}
  \ket{\Psi} \simeq \Psi\qty(r, \phi, \theta) = R\qty(r)Y_{l}^m\qty(\phi, \theta),
\end{equation}
where $R\qty(r)$ is the radial part, and $Y_{l}^m\qty(\theta, \phi)$ are the spherical
harmonics. What we need to solve then is the radial equation
\begin{equation}
  \label{eq:radeq}
  \qty(\frac{-\hbar^2}{2m}\dv[2]{}{r} + V\qty(r) + \frac{l\qty(l+1)}{r^2})u\qty(r) = Eu\qty(r).
\end{equation}
(See the appendix for more details on the wavefunction and the radial eq.) Here $l$ is the orbital momentum, $u\qty(r) = rR\qty(r)$, and $E$ are the
eigenvalues of $\Psi\qty(r, \theta, \phi)$.
We will assume our electron has no orbital momentum ($l = 0$), and scale eq. \ref{eq:radeq}
by substituting $\rho = \frac{r}{\alpha}$ and $\lambda = \frac{E}{\epsilon}$, inserting $V\qty(r)$, and get
  \begin{equation}
    -\frac{\hbar^2}{2m\alpha^2}\qty(\dv[2]{}{\rho} - \frac{m^2\omega^2 \alpha^4}{\hbar^2} \rho^2)u\qty(\rho) = \epsilon\lambda u\qty(\rho),
  \end{equation}
where we can define a natural energy scale $\epsilon = \frac{\hbar^2}{2m\alpha^2}$,
and a natural length scale $\alpha = \sqrt{\frac{\hbar}{m\omega}}$, yielding the
dimensionless equation
  \begin{equation}
    \qty(-\dv[2]{}{\rho} + \rho^2)u\qty(\rho) = \lambda u\qty(\rho).
  \end{equation}
Discretising as in eq. \ref{eq:disc1}, the equation becomes
  \begin{equation}
  \label{eq:disc2}
    -\frac{v_{i+1} - 2v_i + v_{i-1}}{h^2} + V_i v_i = \lambda v_i,
  \end{equation}
where $V_i = \rho_{i}^2=\qty(\rho_0 + ih)^2$. Enforcing the Dirichlet boundary conditions,
we see that \ref{eq:disc2} can be written on matrix form as
  \begin{equation}
    A\vb{v} = \lambda \vb{v}.
  \end{equation}
Here
  \begin{equation}
    A =
      \mqty[d_1^e & a & 0 & \hdots & \hdots & 0 \\
            a & d_2^e & a & 0 & \hdots & 0 \\
            0 & a & d_3^e & a & \hdots & 0 \\
            \vdots & \ddots & \ddots & \ddots & \ddots & \vdots \\
            0 & \hdots & \ddots & a & d_{n-1}^e & a \\
            0 & \hdots & \hdots & 0 & a & d_n^e],
  \end{equation}
where $d_i^e = \frac{2}{h^2} + V_i$ and $a$ is as before. The analytical eigenvalues
are here given by:

\subsection{Two electrons in an harmonic oscillator potential}
\label{sec:qmdots}

We will now consider the problem of two electrons in the aforementioned potential.
As the electrons are interacting, we will have to modify the radial equation by adding the Coloumb interaction term
  \begin{equation}
    V\qty(r_1, r_2) = \frac{\beta e^2}{\abs{\vb{r}_1 - \vb{r}_2}},
  \end{equation}
where $\beta e^2 = 1.44 eVnm$, $e$ is the electron charge, and $\vb{r}_1$ and $\vb{r}_2$ are the
positions of electron 1 and 2 respectively.
To get the modified radial equation on the form of eq. \ref{eq:eigenspecial}, we use the relative distance
$\vb{r} \equiv \vb{r}_1 - \vb{r}_2$, center of mass $\vb{R} \equiv \frac{1}{2}\qty(\vb{r}_1 + \vb{r}_2)$
reference system. We will further only consider the radial solution of the relative distance, and not the center of mass.
Using the same scaling parameters $\alpha$, $\epsilon$, as before
(see appendix for the full procedure), we obtain the equation:
  \begin{equation}
    \qty(-\dv[2]{}{\rho} + \omega_r^2\rho^2 + \frac{1}{\rho})\psi\qty(r) = \lambda \psi\qty(r).
  \end{equation}
Here $\rho = \frac{r}{\alpha}$, $\omega_r^2 = \frac{1}{4}\frac{m^2\omega^2}{\hbar^2}\alpha^4$, and
$\psi\qty(r)$ is the relative distance part of the radial solution. Discretising
as usual, and enforcing the Dirichlet conditions, we get our equation on matrix form
  \[A\vb{v} = \lambda \vb{v},
  \]
where the diagonal elements are now given by $d_{i}^{2e} = \frac{2}{h^2} + \omega_r^2 \rho^2 + \frac{1}{\rho}$.

\subsection{Algorithm: Jacobi's method}
\label{sec:jacobi}
The strategy of Jacobi's method is to perform a series of similarity transformations on the matrix A in order to diagonalize the matrix. We say that matrix B is similar to A if
 \begin{equation}
	B = S^{-1}AS, 
 \end{equation}
and that the transformation from A to B is a similarity transformation. If S is a real orthogonal matrix we have that 
 \begin{equation}\label{eq:sim}
	 S^{-1}=S^T \quad \text{and} \quad B = S^TAS.
 \end{equation}
Since A is a real symmetric matrix there exists a real orthogonal matrix P such that 
 \begin{equation}
     D=P^TAP
 \end{equation}
is a diagonal matrix. Furthermore, the entries along the diagonal of D is the eigenvalues of A, and the column vectors of P are the corresponding eigenvectors. Since the matrix product of two orthogonal matrices is another orthogonal matrix, we can preform a series of similarity transformations until we (ideally) get a diagonal matrix. That is 
\begin{equation}
S_j^TS_{j-1}^T\cdots S_1^TAS_1 \cdots S_{j-1}S_j = D
\end{equation}
where $S_i, \quad i = 1, \ldots, j $ are real orthogonal matrices. In Jacobi's method, the matrix $S_i$ is on the form

\begin{equation}\label{eq:S}
S =
\mqty[s_{11} & \hdots & s_{1n} \\
\vdots & \ddots & \vdots \\
s_{n1} & \hdots & s_{nn} ],
\end{equation}
where 
\begin{equation}
\begin{split}
&s_{kk}=s_{ll} = \cos \theta \\
&s_{lk} = -s_{kl} = \sin \theta \\
&s_{ii} =  1, \quad i \neq k,l 
\end{split}
\end{equation}
and elsewhere zero. The matrix $S_i$ is a rotational matrix, which performs a plane rotation around an angle $\theta$ in the n-dimensional Euclidean space. For the full transformation expressions, see the appendix.

 The idea now is to set $k$ and $l$ such that $a_{kl}=a_{lk}$ is the largest (not considering the sign) non-diagonal element in A, and choose $\theta$ so that $b_{kl}=0$. See the appendix for the expression for $\theta$. 
Notice that when rotating A such that $a_{kl}$ is set to zero, other matrix elements previously equal to zero may change. However, the frobenius norm of an orthogonal transformation is always
preserved, so no elements will blow up.
Using the same procedure on the new matrix, we repeat until all non-diagonal element are essentially zero. Since we want the eigenvectors as well, we need to compute $S_1 \cdots S_{j-1}S_j$ continuously. This gives us the following algorithm
\begin{algorithm}[h!]
	\SetAlgoLined
	initialise $A$, $S$\;
	\While{$max > tol$ }{
		find $max,k,l$\;
		compute $\tau, \sin\theta, \cos\theta$\;
		rotate $A$\;
		update $S$\;
	}
\end{algorithm} 
 
\noindent Now, the diagonal elements of $A$ and the column vectors of $S$ are the eigenvalues and the corresponding eigenvectors, respectively. 

\subsection{Error}
\label{sec:error}

When the analytical solution is known, we define the error in the eigenvalue as

\begin{equation}
\epsilon_i=\left(\left|\frac{\lambda_i-\lambda_i^{comp}}{\lambda_i}\right|\right)
\end{equation}
where $\lambda_i$ is the analytical value and $\lambda_i^{comp}$ is the computed value.
\subsection{Comparing CPU time}
To get an ide how the algorithm perform we compare the CPU time for our implementation of Jacobi's method and Armadillo's \texttt{eig\_sym} function applied at the buckling beam problem. In both cases we time the computation of both the eigenvalues and the eigenvectors. The mean time of ten runs is recorded for each n.  
\subsection{Programming technicalities}
\label{sec:progtech}

All our programs are written in C++, using python3.6 to produce figures and tables.
We use armadillo with LAPACK to define matrices and vectors, as well as comparing
LAPACK's eigenvalue solver with the Jacobi algorithm. All our code can be found in
a github repository "FYS3150" by janadr\footnote{https://github.com/janadr/FYS3150/tree/master/prosjekt2}.
