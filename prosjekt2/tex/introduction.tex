\section{Introduction}
\label{sec:introduction}

A great variety of problems in the physical sciences can be represented as
eigenvalue problems, which generally takes the form:
  \begin{equation}
    \label{eq:eigengeneral}
    O\vb{v} = \lambda \vb{v},
  \end{equation}
where $O$ is an operator, $\lambda$ is an eigenvalue, and $\vb{v}$ is an eigenvector.
Such problems can easily be solved in terms of linear algebra, and is therefore
of great use in simplifying complicated problems, as well as providing a framework for creating efficient
algorithms.
In particular, we will look at the motion of a fixed buckling beam, and show that this is numerically
identical to the motion of quantum dots. Specifically, we will solve the following
eigenvalue problem:
  \begin{equation}
  \label{eq:eigenspecial}
    \qty(\dv[2]{}{x} + f\qty(x))u\qty(x) = \lambda u\qty(x).
  \end{equation}
Starting with the methods section, we will present the buckling beam problem, as well
as the quantum dot problem, scaling and generalising them to the form of eq. \ref{eq:eigengeneral}.
Using different numerical methods, we will use unit testing to make sure that each
algorithm is implemented correctly.
Moving on to the results section, we will present the efficiency and error of each
algorithm in terms of CPU-time and relative error respectively. Then finally,
in the discussion section, we will compare the different methods, and look at
more possiblities for problems that can be solved using the same general algorithms.
