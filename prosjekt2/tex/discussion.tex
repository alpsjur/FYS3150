\section{Discussion}
\label{sec:discussion}

As we have seen in Figure \ref{fig:CPUtime}, Jacobi's method is quite time consuming for large $n$, compared to Armadillos function \texttt{eig\_sym}.  Our implementation of Jacobi's method is more efficient than Armadillo's function for $n$ smaller than about 20, but for larger $n$ the roles change. Estimating the slope, we get that the CPU-time is proportional to approx. $n^4$. The number of iterations is proportional to approx. $n^2$, so CPU-time per iteration must increase as $n^2$. In our case, $n$ is the resolution in the discretization of the displacement function for the buckling beam, and of the wave function for the electron(s). So large $n$ is usually preferred, and therefore Armadillo may be a better choice.

The relative error, on the other hand, converges quickly, as shown in Figure \ref{fig:error}. The relative error is smallest for small $n$, and stabilises quickly to $\sim10^{-5.77}$ for $n$ larger than 25.

So Jacobi's method gives a good approximation, even though it can be slow for large $n$.

Looking back at Figure \ref{fig:quantum_dots_nointer}, the effects of increasing
$\omega_r$ can be understood in the perspective that $\omega_r$ alters the strength
of the potential. So for larger $\omega_r$, the potential is stronger, and the electrons are more likely to be measured
closer to eachother. As for the narrowing of the wave function,
we see that the energy eigenvalues increase because of the greater potential energy
of the electrons. We see then that as $\omega_r$ increases, the electrons become
more and more trapped, and they are therefore less likely to be spread out.

Comparing the case of no Coloumb interaction, and the case with, we see that
the wave function is shifted to the right for all $\omega_r$, but at the greatest
for the lowest $\omega_r$. This means that the effect is most prominent for weaker
potentials. At close range, the Coloumb force $\frac{1}{\rho}$ will dominate,
and the electrons will repel eachother heavily. The spread in $\psi_0^2$ will thus increase
until there is a new equillibrium between the two. This makes it so that it is more likely to
measure the electrons further away from eachother. The fact that the energy eigenvalues
increase more and more for increasing $\omega_r$ is because the stronger harmonic oscillator
potential makes the Coloumb interaction store more potential energy, as $\rho$ must be smaller
to stay in equillibrium.

In conclusion, we have implemented and benchmarked Jacobi's algorithm for solving eigenvalue problems.
Here, we find that our implementation is generally slow compared to armadillo's built-in function $eig_sym()$,
but that the error stabilises for increasing $n$ very quickly.
We applied the algorithm on three different physical systems, where we used the case
of the buckling beam for benchmarking, and the case of the single electron for
showing similarity between the two. Our analysis was therefore focused on the problem
of two interacting electrons, where we find that a greater harmonic oscillator potential
results in a narrower wave function as well as higher energy eigenvalues. Including
the Coloumb interaction between the electrons, we find that they are more likely to be
found further apart, and that the interaction amplifies the effect of increasing
the potential energy of the electrons.
