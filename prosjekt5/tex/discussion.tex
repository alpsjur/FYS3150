\section{Discussion}
\label{sec:discussion}

Looking back at the Hovmöller diagrams, we find that Figure \ref{fig:hovmollerSinePeriodic} corresponds well with the analytical expression in eq. \ref{eq:periodic sol} as shown in the figure is the very distinct Hovmöller diagram for a cosine wave with a constant amplitude of 1. In the bounded case (Figure \ref{fig:hovmollerSineBounded}) we find that the amplitude still is 1, compared to the equilibrium line. Here, the numerical solution does not match the analytical solution. The bounded solution with an initial Gaussian wave, however, seems to match better with the analytical solution, since the amplitude increases toward the centre of the domain. The Gaussian initial wave does, on the other hand, not match well with the analytical periodic solution. This leads to the impression that an initial sine wave is better for the periodic domain, while a Gaussian wave is prefered for the bounded domain. Both waves are easterlies, which is as expected from the analytical negative dispersion relation (eq. \ref{eq:phasespeed}), and correlates with Rossby's original observations \citep{rossby}.

As for the gaussian waves, we have no analytical solutions, but find that 