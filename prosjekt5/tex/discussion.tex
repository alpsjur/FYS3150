\section{Discussion}
\label{sec:discussion}

Looking back at the Hovmöller diagrams, we find that Figure \ref{fig:hovmollerSinePeriodic} corresponds well with the analytical expression in eq. \ref{eq:periodic sol} as shown in the figure is the very distinct Hovmöller diagram for a cosine wave with a constant amplitude of 1. In the bounded case (Figure \ref{fig:hovmollerSineBounded}) we find that the amplitude is still 1, as expected from the analytical result, taking into account that the equilibrium line oscillates around the initial equilibrium line, also with an amplitude of 1. The Gaussian initial condition does not yield results comparable to the analytical expression. However, both waves are easterlies, which is as expected from the analytical negative dispersion relation (eq. \ref{eq:phasespeed}), and correlates with Rossby's original observations \citep{rossby}. 

A deeper analysis could be performed for measuring the stability criterion for the two schemes. Figure \ref{fig:compare} only shows a snapshot where the explicit schemes fails. A more critical analysis would require finding the stability as a function of $\Delta t$.

Looking back at the Hovmöller diagrams, we find that Figure \ref{fig:hovmollerSinePeriodic} corresponds well with the analytical expression in eq. \ref{eq:periodic sol} as shown in the figure is the very distinct Hovmöller diagram for a cosine wave with a constant amplitude of 1. In the bounded case (Figure \ref{fig:hovmollerSineBounded}) we find that the amplitude still is 1, compared to the equilibrium line. Here, the numerical solution does not match the analytical solution. The bounded solution with an initial Gaussian wave, however, seems to match better with the analytical solution, since the amplitude increases toward the centre of the domain. The Gaussian initial wave does, on the other hand, not match well with the analytical periodic solution. This leads to the impression that an initial sine wave is better for the periodic domain, while a Gaussian wave is prefered for the bounded domain. Both waves are easterlies, which is as expected from the analytical negative dispersion relation (eq. \ref{eq:phasespeed}), and correlates with Rossby's original observations \citep{rossby}.

As for the gaussian waves, we have no analytical solutions, but find that they still propagate westward (as seen in Figure \ref{fig:hovmollerGaussianPeriodic} and \ref{fig:hovmollerGaussianBounded}). The results we see for varying $\sigma$ implies that the distribution of anti-nodes of the gaussian wave is preserved throughout the motion, seeing as they become more concentrated for a narrower wave, and less concentrated for a wider wave. While we couldn't find the phase velocity, we can qualitatively (and very roughly) see that the phase velocity seem to decrease in the bounded domain, which is consistent with the sine wave.

In two dimensions we only present the sine waves, as the gaussian waves yielded unphysical results. From Figure \ref{fig:periodicsine2d} and \ref{fig:boundedsine2d} we could not discern the direction of propagation. However, referring to the code for animating the time evolution found in the GitHub repository (see Theory), it can be seen that the wave is travelling to the west. This being the case for both types of boundary conditions indicates that the one-dimensional problem is a representation for the two-dimensional problem, as there is very little, if no movement zonally.

