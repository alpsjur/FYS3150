\section{Introduction}
\label{sec:introduction}

We encounter wave phenomena everywhere in the natural sciences. From quantum mechanics to oceanography, we find that be it the motion of a particle or the ocean, we require knowledge of wave-like behaviour to solve the problem. In quantum mechanics, a particle's wave function is described by a complex-valued diffusion equation, the Schrödinger equation, while in oceanography, we can describe ocean waves using the wave equation,
	\begin{equation}
	\label{eq:wave}
	\pdv[2]{u}{x} = \pdv[2]{u}{t},
	\end{equation}
where x and t denote the spatial and temporal coordinates, respectively. The wave equation will be the topic of this paper, in particular, we will model Rossby waves, first identified by \citet{rossby}. These are inertial, planetary waves in the Earth's atmosphere and ocean which motions contribute to extreme weather \citep{mann2017influence}, might drive the El-Ninõ southern oscillation (ENSO) \citep{bosc2008observed}, and is also produced by ENSO, see \citet{battisti1989role}. While we hope the reader appreciate the wide range of phenomena related to these waves, our article is a numerical study of the waves isolated from other processes. We therefore begin by describing fundamental theory of waves and partial differential equations in the Theory section, present our algorithm and the technicalities relating to its implementation. In the Results section, we present our data as figures, before discussing their implications in the Discussion section. Concluding our paper, we present our final thoughts on the topic of simulating Rossby waves.