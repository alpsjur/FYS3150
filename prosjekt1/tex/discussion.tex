\section{Discussion}
\label{sec:discussion}

Having presented the different methods and their relative efficiency and
accuracy, we would like to discuss their uses as well as their boundaries.
We see that for our problem of solving eq. \ref{eq:gendiff}, we could construct
a very specialised and efficient algorithm because we used a 2nd order
central approximation to the 2nd derivative. This let us reduce the number of
flops by a half, and our computer should, with an optimised CPU, only have to do
half the work. In terms of accuracy, we found no significant difference between
the algorithms, and so this also means that we get the most accuracy per CPU time
with our specialised algorithm up until n=$10^{-6}$ (see \ref{fig:compare}).
Thus for our problem of a linear second order differential equation with
Dirichlet boundaries, this is clearly the most advantageous algorithm.

However, while the specialised algorithm is good at solving eq. \ref{eq:gendiff},
it is, as our naming suggests, not very general. There exists a wide variety of
problems that can be expressed in the form of our matrix equation, eq.
\ref{eq:matrixeq}, where the matrix A takes different forms.
