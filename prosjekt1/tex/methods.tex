\section{Methods}
\label{sec:methods}

We would like to solve eq. \ref{eq:poisson} numerically. Generalising the
equation, we get
  \begin{equation}
    -\dv[2]{u}{x} = f\qty(x),
  \end{equation}
where we have assumed that $\rho \propto \frac{1}{r}e^{-r}$ and let $ r
\rightarrow x$, $\phi \rightarrow u$.
Summing the backward and forward Taylor expansions of $u(x)$ and discretising
the equation for $n$ integration points, we get:
  \begin{equation}
  \label{eq:approx}
    \frac{v_{i+1} - 2v_i + v_{i-1}}{h^2} + \order{h^2} = f_i,
  \end{equation}
where $h = \frac{1}{n+1}$. Using the Dirichlet boundary conditions
$v_0 = v_{n+1} = 0$ and only considering $x\in\qty(0, 1)$, we can rewrite the
equation as a set of linear equations, represented as the following matrix
equation:
  \begin{equation}
  \label{eq:matrixeq}
    A\vb{v} = \vb{d},
  \end{equation}
where
  \[A =
    \mqty[2 & -1 & 0 & \dots & \dots & 0 \\
          -1 & 2 & -1 & 0 & \dots & 0 \\
          0 & -1 & 2 & -1 & \dots & 0 \\
          \vdots & \ddots & \ddots & \ddots & \ddots & \vdots \\
          0 & \dots & \ddots & -1 & 2 & -1 \\
          0 & \dots & \dots & 0 & -1 & 2],
  \]
is a tridiagonal matrix and $d_i = h^{2}f_i$.
Having the equation in this form, we can use linear algebra to solve the set of
linear equations, and obtain the 2nd derivative of $u\qty(x)$.


\subsection{Gaussian elimination}
\label{sec:gaussian}

The fact that $A$ is tridiagonal, means that we can develop a general algorithm
to solve the equations by the method of Gaussian elimination. If we generalise
our matrix
  \[A =
    \mqty[b_1 & c_1 & 0 & \dots & \dots & 0 \\
          a_1 & b_2 & c_2 & 0 & \dots & 0 \\
          0 & a_2 & b_3 & c_3 & \dots & 0 \\
          \vdots & \ddots & \ddots & \ddots & \ddots & \vdots \\
          0 & \dots & \ddots & a_{n-2} & b_{n-1} & c_{n-1} \\
          0 & \dots & \dots & 0 & a_{n-1} & b_n],
  \]
and row reduce by subtracting the first row times $\frac{a_1}{b_1}$ from the
second row ($I - \frac{a_1}{b_1}II)$), we see that the new diagonal element is
given by
  \[\widetilde{b_2} \equiv b_2 - \frac{a_1}{b_1}c_1.\]
Considering eq. \ref{eq:matrixeq} then, the corresponding change on the left
hand side is given by
  \[\widetilde{d_2} \equiv d_2 - \frac{a_1}{b_1}d_1.\]
Performing the same operation on the next row
($III - \frac{a_2}{\widetilde{b_2}}II$), we get
  \[\widetilde{b_3} \equiv b_3 - \frac{a_2}{\widetilde{b_2}}c_2, \quad
  \widetilde{d_3} \equiv d_3 - \frac{a_2}{\widetilde{b_2}}\widetilde{d_2}.
  \]
