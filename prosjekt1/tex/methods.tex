\section{Methods}
\label{sec:methods}

We would like to solve eq. \ref{eq:poisson} numerically. Generalising the
equation, we get
  \begin{equation} \label{eq:gendiff}
    -\dv[2]{u}{x} = f\qty(x),
  \end{equation}
where we have assumed that $\rho \propto \frac{1}{r}e^{-r}$ and let $ r
\rightarrow x$, $\phi \rightarrow u$.
Summing the backward and forward Taylor expansions of $u(x)$ and discretising
the equation for $n$ integration points, we get:
  \begin{equation}
  \label{eq:approx}
    \frac{v_{i+1} - 2v_i + v_{i-1}}{h^2} + \order{h^2} = f_i,
  \end{equation}
where $h = \frac{1}{n+1}$. Using the Dirichlet boundary conditions
$v_0 = v_{n+1} = 0$ and only considering $x\in\qty(0, 1)$, we can rewrite the
equation as a set of linear equations, represented as the following matrix
equation:
  \begin{equation}
  \label{eq:matrixeq}
    A\vb{v} = \vb{d},
  \end{equation}
where
  \[A =
    \mqty[2 & -1 & 0 & \hdots & \hdots & 0 \\
          -1 & 2 & -1 & 0 & \hdots & 0 \\
          0 & -1 & 2 & -1 & \hdots & 0 \\
          \vdots & \ddots & \ddots & \ddots & \ddots & \vdots \\
          0 & \hdots & \ddots & -1 & 2 & -1 \\
          0 & \hdots & \hdots & 0 & -1 & 2],
  \]
is a tridiagonal matrix and $d_i = h^2f_i$.
Having the equation in this form, we can use linear algebra to solve the set of
linear equations, and obtain the 2nd derivative of $u\qty(x)$.


\subsection{Gaussian elimination}
\label{sec:gaussian}

The fact that $A$ is tridiagonal, means that we can develop a general algorithm
to solve the equations by the method of Gaussian elimination. If we generalise
our matrix
  \[A =
    \mqty[b_1 & c_1 & 0 & \hdots & \hdots & 0 \\
          a_1 & b_2 & c_2 & 0 & \hdots & 0 \\
          0 & a_2 & b_3 & c_3 & \hdots & 0 \\
          \vdots & \ddots & \ddots & \ddots & \ddots & \vdots \\
          0 & \hdots & \ddots & a_{n-2} & b_{n-1} & c_{n-1} \\
          0 & \hdots & \hdots & 0 & a_{n-1} & b_n],
  \]
and row reduce $A$ to become upper-triangular (see appendix A.)
This yields us the following relations:
  \begin{equation}  \label{eq:gaussian}
    \widetilde{b_i} = b_i - \frac{a_{i-1}}{\tilde{b}_{i-1}}c_{i-1}, \quad
    \widetilde{d_i} = d_i - \frac{a_{i-1}}{\tilde{b}_{i-1}}\tilde{d}_{i-1},
  \end{equation}
where $\widetilde{b_1} \equiv b_1$ and $\widetilde{d_1} \equiv d_1$.
Performing the matrix-vector multiplication in eq. \ref{eq:matrixeq}, we get
the following relations from the second-to-last and last row:
  \begin{equation}
    v_i = \frac{d_i - c_iv_{i+1}}{\widetilde{b_i}}, \quad
    v_n = \frac{\widetilde{d}_n}{\widetilde{b_n}}.
  \end{equation}
With these expressions, we can construct our general algorithm as follows:
  \begin{algorithm}[h!]
    \SetAlgoLined
    initialise $\vb{v}$, $\vb{a}$, $\vb{b}$, $\vb{c}$, $\vb{d}$\;
    \For{$i = 2, \dots, n$}{
      $b_i = b_i - \frac{a_{i-1}}{b_{i-1}}c_{i-1}$\;
      $d_i = d_i - \frac{a_{i-1}}{b_{i-1}}c_{i-1}$\;
    }
    $v_n = \frac{d_n}{b_n}$\;
    \For{$i = n-1, \dots, 1$}{
      $v_i = \frac{d_i - c_i v_{i+1}}{b_i}$\;
    }
  \end{algorithm}

This algorithm has a total of 8n FLOPS.

If we now take into account that our matrix has only twos along the diagonal, and -1 above and below the diagonal, we can make a specialised algorithm. Plugging in theses values in the expression for $\widetilde{b_i}$ in equation \ref{eq:gaussian} gives us the following:

	\begin{equation}
	\widetilde{b_i}=2-\frac{1}{\widetilde{b}_{i-1}}, \quad \widetilde{b}_1=2
	\end{equation}
We can prove by induction that this is the same as:

	\begin{equation}
	\widetilde{b}_i = \frac{i+1}{i}, \quad i \geq 1
	\end{equation}

Since $\widetilde{b_i}$ is only a function of $i$ we can now set up $\mathbf{\widetilde{b}}$ in the initialisation. Plugging in the values in the remaining expressions gives:

	\begin{equation}
	\widetilde{d_i} = d_i + \frac{\widetilde{d}_{i-1}}{\widetilde{b}_{i-1}}, \quad
	v_i = \frac{\widetilde{d_i}+u_{i+1}}{\widetilde{b_i}}
	\end{equation}
where $v_n = \frac{\widetilde{d_n}}{\widetilde{b_n}}$. This gives us the following specialised algorithm, with a total of 4n FLOPS (not counting the initialisation of $\mathbf{\widetilde{b}}$):

	\begin{algorithm}[h!]
	\SetAlgoLined
	initialise $\vb{v}$, $\vb{b}$, $\vb{d}$\;
	\For{$i = 2, \dots, n$}{
		$d_i = d_i + \frac{d_{i-1}}{b_{i-1}}$\;

	}
	$v_n = \frac{d_n}{b_n}$\;
	\For{$i = n-1, \dots, 1$}{
		$v_i = \frac{d_i + v_{i+1}}{b_i}$\;
	}
	\end{algorithm}



\subsection{LU-decomposition}
\label{sec:LU}

Let A be a square matrix, then we can decompose it in the following form
  \begin{equation}
    A = LU,
  \end{equation}
where L is a lower triangular matrix and U is a upper triangular matrix. For
further conditions on A, see \cite{LUref}. This lets us rewrite eq.
\ref{eq:matrixeq} in the form:
  \begin{equation}
    \label{eq:LUvd}
    \qty(LU)v = d,
  \end{equation}
where we let
  \begin{equation}
    \label{eq:yUv}
    y = Uv,
  \end{equation}
  and rewrite eq. \ref{eq:LUvd} to
  \begin{equation}
    \label{eq:Lyd}
    Ly = d.
  \end{equation}
To solve for $v$ then we need to first solve eqs. \ref{eq:Lyd} for $y$ and then
substitute $y$ into eq. \ref{eq:yUv} where we can then solve for $v$, yielding
us the solution to eq. \ref{eq:matrixeq}.

This then is our general LU-decomposition method, and it has around
$\frac{2}{3}n^3$ flops. We could elaborate further on the finite difference
details hidden away in the notation, but this algorithm is implemented in
armadillo (see the Programming technicalities section), and so most of these details
aren't handled by us.

\subsection{Error}
\label{sec:error}

We define the error for each mesh point as:

	\begin{equation}
	\epsilon_i=log_{10}\left(\left|\frac{v_i-u_i}{u_i}\right|\right)
	\end{equation}
where $u_i=u(x_i)$ is the analytical solution given by $u(x) = 1-(1-e^{-10})x-e^{-10x}$.

\subsection{Programming technicalities}
\label{sec:proglang}

We implement our algorithms and methods in C++. For our Gaussian elimination
algorithm we use only the standard library to declare and operate arrays.
For the LU decomposition algorithm we use the armadillo library with LAPACK to
declare matrices and perform the LU-decomposition as well as solving eqs.
\ref{eq:Lyd} and \ref{eq:yUv}.
