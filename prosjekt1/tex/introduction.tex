\section{Introduction}
\label{sec:introduction}

Physics is a field concerned with the behaviour of nature, and nature is
everchanging. It is therefore no surprise that differential equations appear
everywhere in physics. From global climate dynamics to statistical mechanics,
what we find is that differential equations, often many and coupled, are
required to explain or model the phenomena. For such large models, efficiency
is important, as we would, for example, like to have timely weather forecasts.
One way to make a model more efficient is by using an efficient algorithm for
solving differential equations.

In this article, we compare two different numerical methods of solving linear
second-order differential equations with the Dirichlet boundary conditions.
To do this, we will solve the one-dimensional Poisson's equation:
  \begin{equation}
  \label{eq:poisson}
    \dv[2]{\phi}{r} = -4\pi r \rho\qty(r).
  \end{equation}
In the methods section, we develop an approximation for the
2nd derivative to the 2nd order. We will then solve eq. \ref{eq:poisson}
numerically, using gaussian elimination and lower-upper decomposition. As for
the former, we will further specialise it to solve eq. \ref{eq:poisson} more
efficiently. Next, in the results section we present the comparison
between our numerical solutions and the analytical solution, as well as the error.
We then compare the efficiency of all three algorithms, and finally, in the
discussion section we will consider the advantages and disadvantages
of each algorithm, and discuss their uses.
