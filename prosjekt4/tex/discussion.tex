\section{Discussion}
\label{sec:discussion}
The relative error in our algorithm, as shown in Figure \ref{fig:compare}, behave as expected in case of the mean energy. As for the absolute magnetisation, the relative error remaining approximately constant might be due to a systematic error.

Determining the amount of Monte Carlo cycles required to reach a equilibrium state (see Figure \ref{fig:MC}) was important to get a good sample, as we are only interested in the equilibrium state. Including the cycles before this point would increase variance in our results and make them more inaccurate as a consequence. From the figures we determined that equilibrium was obtained after around 5000 Monte Carlo Cycles for both ordered and unordered lattices in the case of the mean energy. For the absolute magnetisation, the thermalisation time seem to be longer, and we determine equilibrium to be at around 15000 cycles. In our program we set this to be 25000 cycles. This is to make certain that equilibrium is reached for all cases, and because it does not alter our results by much as the magnitude of Monte Carlo cycles is of order $\order{6} - \order{7}$. As for the probability distribution function (pdf) of our Metropolis algorithm we see from Figure \ref{fig:probabilitydist} for $T=2.4$ a pdf that resembles the gamma distribution, due to the tail left in the figure. When $T=1.0$ there seems to be no apparent pdf behind the algorithm. This implies that there are less available microstates for lower temperatures. Comparing this with the obtained values of the expected energy and standard deviation (see Table \ref{table:energystd}) we see agreement in the standard deviation being quite low for $T=1.0$ and high for $T=2.4$, fitting the shape of their distributions. This behaviour also fits well with the observed amount of accepted states (see Figure \ref{fig:acceptance}) as there are more available states in the $T=2.4$ case.

We determine the phase transition to be at the maxima observed in the heat capacity (see \ref{fig:heatcap}) as there are many indications of transitioning. The susceptibility function increases quite rapidly for increasing $L$ in the temperature region (see Figure \ref{fig:susceptibility001}), implying divergence. Looking back at Table \ref{table:Tmax} we see that all obtained critical temperatures using this point is within a standard deviation of the estimated analytical value of 2.269 (see eq.\ref{eq:analyticT}). An issue is that the standard deviation is relatively large. Comparing with similar analyses, we are led to believe that this might be caused by an error in implementation when producing the $L= 40, 60, 80, 100$ data, as we neglected initialising the Ising lattice for each temperature step. Nevertheless, we conclude that the Metropolis algorithm yields good result when used to simulate the Ising model.
