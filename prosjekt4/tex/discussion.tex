\section{Discussion}
\label{sec:discussion}
The relative error in our algorithm, as shown in Figure \ref{fig:compare}, behave as expected in case of the mean energy. As for the absolute magnetisation, the relative error remaining approximately constant might be due to a systematic error, a disadvantage of the model itself.

Determining the amount of Monte Carlo cycles required to reach a equilibrium state (see Figure \ref{fig:MC}) was important to get a good sample, as we are only interested in the equilibrium state. Including the cycles before this point would increase variance in our results and make them more inaccurate as a consequence. From the figures we determined that equilibrium was obtained after around 1000 Monte Carlo Cycles for both ordered and unordered lattices. In our program however, we set this to be 25000 cycles. This is to make certain that equilibrium is reached for all cases, and because it does not alter our results by much as the magnitude of Monte Carlo cycles is of order $\order{6} - \order{7}$. As for the probability distribution function (pdf) of our Metropolis algorithm we see from Figure \ref{fig:probabilitydist} for $T=2.4$ a pdf that resembles the gamma distribution, due to the tail left in the figure. When $T=1.0$ there seems to be no apparent pdf behind the algorithm. This implies that there are less available microstates for lower temperatures.

The energy of the lattices increase for higher temperatures as expected from an equilibrium system (see Figure \ref{fig:energy001}) and the low variance between lattice sizes is likely due to the energies being calculated only from neighbours. A decrease in absolute magnetisation as seen in Figure \ref{fig:absmag} is also expected as the 
