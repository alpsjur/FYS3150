\section{Methods}
\label{sec:methods}

\subsection{Ising model}
\label{sec:ising}
To study phase transitions emerging from statistical mechanics, we will use the Ising model. In the Ising model we consider a lattice of interacting spins which are influenced by eachother and an magnetic field $B$. In general then, we can express the Ising model as an Hamiltonian:
  \begin{equation}
  \label{eq:isingH}
    H = -\sum_{\expval{i\, j}}J_{ij}s_i s_j - \mu\sum_j B_js_j,
  \end{equation}
where $\expval{i\,j}$ denotes a multiplication only over neighbours, $J_{ij}$ is a coupling parameter, $\mu$ is the magnetic moment, and $B_j$ is a component of an external magnetic field. As our interest primarily lay in identifying phase transitions, we will simplify the model. We assume that the coupling parameter $J_{ij}$ is the same for all spins and positive, i.e. the interaction is ferromagnetic, and that the lattice is not affected by an external magnetic field. Our model then becomes
  \begin{equation}
  \label{eq:ising}
    E \equiv H = -J\sum_{\expval{i\, j}}s_i s_j.
  \end{equation}
From eq. \ref{eq:ising} we can get the mean energy of the system, of which we can use to calculate the specific heat capacity:
  \begin{equation}
    \label{eq:energy}
    \expval{c_V} = \pdv{\expval{E}}{T} = \frac{\beta}{T}\pdv{\expval{E}}{\beta}.
  \end{equation}
Here $\beta = \frac{1}{kT}$ (see eq. \ref{eq:partition}) is the Boltzmann factor relating the energy of the system with the temperature. Furthermore, we can obtain the magnetisation of the system
  \begin{equation}
  \label{eq:magnet}
    M = \sum_i s_i,
  \end{equation}
and use this to construct the suceptibility

\subsection{Metropolis algorithm}
\label{sec:metropolis}

\subsection{Implementation}
\label{sec:implementation}
