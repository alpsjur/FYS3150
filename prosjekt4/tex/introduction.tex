\section{Introduction}
\label{sec:introduction}

Statistical mechanics aims to describe thermodynamics from a quantum mechanical point of view.
By making a system large enough, we should approach the thermodynamic limit where classical behaviour emerges. A concept which relates quantities such as the total energy, heat capacity, entropy, and pressure, is the partition function
  \begin{equation}
    \label{eq:partition}
    Z = \sum_s e^{-\frac{E_s}{kT}},
  \end{equation}
where $k$ is Boltzmann's constant, $T$ is the temperature, and $E_s$ is the energy for each microstate $s$.
The partition function describes the statistical properties of a system in thermal equilibrium.
While useful and interesting theoretically, analytical solutions are rare to come by. In particular,
it is quite difficult to calculate the microstate energies.

The aim of this report is then to find some of these quantities (observables) numerically, using the Metropolis algorithm to see how the energy and magnetisation vary as functions of temperature. We begin by describing the Ising model and Metropolis algorithm in the Methods section, before moving on to results. Here we start by presenting a brief analysis of the Ising model and Metropolis algorithm, and end by giving describing the different observables as functions of temperature. Finally, in the discussion section, we consider the error in our results, discuss the probability distribution of our Metropolis algorithm, and place the data in a physical context looking for phase transitions.
